\documentclass{book}
\usepackage{fancyhdr, graphicx, multicol}

\title{\huge \textbf{The Hog Programming Language}}
\author{Jason Halpern \\ jrh2170 \\ Testing/Validation
        \and Samuel Messing \\ sbm2158 \\ Project Manager
        \and Benjamin Rapaport \\ bar2150 \\ System Architect
        \and Kurry Tran \\ klt2127 \\ System Integrator
        \and Paul Tylkin \\ pt2302 \\ Language Guru}

\begin{document}
\maketitle

\tableofcontents

\chapter{Introduction}
\label{chap:intro}

Use your updated White Paper.

\chapter{Tutorial}
\label{chap:tutor}

Use your updated tutorial.

\chapter{Language Reference Manual}
\label{chap:LRM}

Use your update LRM.

\chapter{Project Plan}
\label{chap:plan}

To be written by Sam.

\begin{itemize}
\item State what process used was used to develop the language and its translator.
\item State the roles and responsibilities of each team member.
\item Include the implementation style sheet used by the team.
\item Show the timeline of what was done and when.
\item Include your project log.
\end{itemize}

\chapter{Language Evolution}
\label{chap:evo}

To be written by Paul.

\begin{itemize}
\item Describe how the language evolved during the implementation and what steps were used to try to maintain the good attributes of the original language proposal.
\item Describe the compiler tools used to create the compiler components.
\item Describe what unusual libraries are used in the compiler.
\item Describe what steps were taken to keep the LRM and the compiler consistent.
\end{itemize}

\chapter{Translator Architecture}
\label{chap:trans}

To be written by Ben.

\begin{itemize}
\item Show the architectural block diagram of translator.
\item Describe the interfaces between the modules.
\item State which modules were written by which team members.
\end{itemize}

\chapter{Development and Run-Time Environment}
\label{chap:environ}

To be written by Kurry.

\begin{itemize}
\item Describe the software development environment used to create the compiler.
\item Show the makefile used to create and test the compiler during development.
\item Describe the run-time environment for the compiler.
\end{itemize}

\chapter{Test Plan}
\label{chap:test}

To be written by Jason.

\begin{itemize}
\item Describe the test methodology used during development.
\item Show programs used to test your translator.
\end{itemize}

\chapter{Conclusions}
\label{chap:concl}

\section{Lessons Learned}
\label{sec:lessons}

\subsection{Jason's Lessons}
\label{sub:jasons-lessons}

To be written by Jason.

\subsection{Sam's Lessons}
\label{sub:sams-lessons}

To be written by Sam.

\subsection{Ben's Lessons}
\label{sub:bens-lessons}

To be written by Ben.

\subsection{Kurry's Lessons}
\label{sub:kurrys-lessons}

To be wrtten by Kurry.

\subsection{Paul's Lessons}
\label{sub:pauls-lessons}

To be written by Paul.

\section{Advice for Other Teams}

Don't take this class.

\section{Suggestions for Instructor}

Don't teach this class.

\appendix

\chapter{Code Listing}

Include a listing of the complete source code with identification of who wrote which module of the compiler. This listing does not have to be included in the paper copy of the final report.

\nocite{*}
\bibliographystyle{acm}
\bibliography{references}

\end{document}
