\documentclass{book}
\usepackage{fancyhdr, multicol}

\title{\huge \textbf{Hog Language Reference}}
\author{Jason Halpern\\ Samuel Messing\\ Kurry Tran \\ 
        Benjamin Rapaport \\ Paul Tylkin}

\begin{document}
\maketitle
\tableofcontents

\chapter{Introduction}
\label{cha:introduction}

As data sets have grown in size, so have the complexities of dealing with them.
For instance, consider wanting to generate counts for all the words in \emph{War
and Peace} by means of distributed computation. Writing in Java and using Hadoop
MapReduce (TM), a simple solution takes over 50 lines of code, as the programmer
is required to specify intermediate objects not directly related to the desired
computation, but required simply to get Hadoop to function properly. Our goal is
to produce a language that can express the same computation in about 10 lines.

\section{The MapReduce Framework}
\label{sec:mapreduce}

With the explosion in the size of datasets that companies have had to manage in
recent years there are many new challenges that they face. Many companies and
organizations have to handle the processing of datasets that are terabytes or even
petabytes in size. The first challenge in this large-scale processing is how to
make sense of all this data. More importantly, how can they process and manipulate
the data in a time efficient and reliable manner. The second challenge is how they
handle this across their distributed systems. Writing distributed, fault tolerant
programs requires a high level of expertise and knowledge of parallel systems.

This was an obvious challenge to the company that has to process more data than any
company on earth, Google. In response to this need, a group of engineers at Google
developed their MapReduce framework in 2004. This high-level framework could be
used for of a variety of tasks, including handling search queries, indexing crawled
documents and processing logs. The software framework was developed to handle
computations on massive datasets that are distributed across hundreds or even
thousands of machines. The motivation behind MapReduce was to create a unified
framework that abstracted away many of the low level details from programmers, so
they would not have to be concerned with how the data is distributed, how the
computation is parallelized and how all this is done in a fault tolerant manner.
MapReduce provides fault tolerance in software rather than in hardware. MapReduce
can handle both unstructured data (files) and structured data (databases), but is
predominantly used with files.

The framework partitions the data across different machines, so that the
computations are initially performed on smaller sets of data distributed across the
cluster. Each cluster has a master node that is responsible for coordinating the
efforts among the slave nodes. Each slave node sends periodic heartbeats to the
master node so it can be aware of progress and failure. In the case of failure, the
master node can reassign tasks to other nodes in the cluster. In conjunction with
the underlying MapReduce framework created at Google, the company also had to build
the distributed Google File System (GFS). This file system ``allows programs to
access files efficiently from any computer, so functions can be mapped
everywhere.'' GFS was designed with the same goals as other distributed file
systems, including ``performance, scalability, reliability and availability.''
Another key aspect of the GFS design is fault tolerance and this is achieved by
treating failures as normal and optimizing for ``huge files that are mostly
appended to and then read.''

Within the framework, a programmer is responsible for writing both Map and Reduce
functions. The map function is applied to all of the input data ``in order to
compute a set of intermediate key/value pairs.'' In the map step, the master node
partitions the input data into smaller problems and distributes them across the
worker nodes in the cluster. This step is applied in parallel to all of the input
that has been partitioned across the cluster. Then, the reduce step is responsible
for collecting all the processed data from the slave nodes and formatting the
output. The reduce function is carried out over all the values that have the same
key such that each key has a single value. which is the answer to the problem
MapReduce is trying to solve. The output is done to files in the distributed file
system.

The use of ``a functional model with user-specified map and reduce operations
allows (Google) to parallelize large computations easily and to use re-execution as
the primary mechanism for fault tolerance.'' A programmer only has to specify the
functions described above and the system handles the rest of the details. The
following diagram illustrates the execution flow of a MapReduce program.

\section{The Hog Language}
\label{sec:hog_language}

Hog is a \textbf{data-oriented}, \textbf{high-level}, scripting language for
creating MapReduce programs. Used alongside Hadoop, Hog enables users to
efficiently carry out \textbf{distributed} computation. Hadoop MapReduce is an
open-source framework for carrying out distributed computation, which is especially
useful for working with large data sets. While it is possible to write code to
carry out computations with Hadoop directly, the framework requires users to
specify low-level details that are often irrelevant to their desired goal.

By building a scripting language on top of Hadoop, we aim to simplify the process.
Built around a \textbf{simple} and highly \textbf{readable} syntax, Hog will let
users focus on what computations they want done, and not how they want to do them.
Hog takes care of all the low-level details required to run computations on
Hadoop’s distributed network. All a user needs to do is tell Hog the location of
their valid Hadoop instance, and Hog will do the rest.

\textbf{TODO:  about restrictions of Hog}

\subsection{Guiding Principles} % (fold)
\label{sub:guiding_principles}

The guiding principles of Hog are:

\begin{itemize}
  \item Anyone can MapReduce
  \item Brevity over verbosity
  \item Preference simplicity over complexity
\end{itemize}


% subsection guiding_princples (end)

\chapter{Program Structure} % (fold)
\label{cha:program_structure}

\section{Overall Structure} % (fold)
\label{sec:overall_structure}

Every Hog program consists of a single source file with a ‘.hog’ extension. This
source file must contain three sections: \tt @Map\rm, and \tt @Reduce\rm, and
\tt @Main \rm and can also include an optional \tt @Functions \rm section. These
sections must be included in the following order:

\begin{verbatim}
@Functions 
.
.
.
@Map <type signature>
.
.
.
@Reduce <type signature>
.
.
.
@Main
\end{verbatim}

% section overall_structure (end)

\section{\tt @Functions\rm} % (fold)
\label{sec:tt_functionsrm}

At the top of every Hog program, the programmer has the option to define functions
in a section called \tt @Functions\rm. Any function defined in this section can be
called from any other section of the program, including \tt @Map\rm, and \tt
@Reduce\rm, and can also be called from other functions defined in the \tt
@Functions \rm section. The section containing the functions begins with the
keyword \tt @Functions \rm on its own line, followed by the function definitions.

Function definitions have the form:

\begin{verbatim}
@Functions
return-type function-name(parameter-list) {
  exprlist
}
\end{verbatim}

The return-type can be any valid Hog type. The rules regarding legal function-names
are identical to those regarding legal variable identifiers. Each parameter in the
parameter-list consists of a valid Hog type followed by the name of the parameter,
which must also follow the naming rules for identifiers. Parameters in the
parameter-list are separated by commas. The @Functions section ends when the next
Hog section begins.

A complete example of an @Functions section:

\begin{verbatim}
@Functions
int min(int a, int b) {
  if (a < b) {
    return a
  }
  else {
    return b
  }
}

list<int> reverseList(list<int> oldList) {
  linst<int> newList()
  for (int i = oldList.len()-1; i >= 0; i--) {
    newList.append(oldList.get(i))
  }
  return newList
}
\end{verbatim}

Function names can be overloaded as long as the function definitions have different
signatures (i.e. parameter lists different in types and/or length). Additionally,
user-defined functions can make reference to other user-defined functions.

% section tt_functionsrm (end)

\section{\tt @Map \rm} % (fold)
\label{sec:tt_map_rm}

The map function in a MapReduce program takes as input key-value pairs, performs
the appropriate calculations and procedures, and emits intermediate key-value pairs
as output. Any given input pair may map to zero, one, or multiple output pairs. The
\tt @map \rm section defines the code for the map function.

The \tt @map \rm header must be followed by the signature of the map function, and
then the body of the map function as follows:

\begin{verbatim}
@Map (key-type key-name, value-type value-name) -> (key-type, value-type) {
.
.
.
}
\end{verbatim}

The first key-value pair defines the key and value types that form the input to the
map function, and names these values so that they can be referred to in the body of
the function. This is followed by an arrow and another key-value pair, defining the
types of the output of the map function. Note that the output signature is
\emph{unnamed}, as these pairs are only produces at the end of the map function
(\textbf{@All: mention something about emit() being similar to return() in that all
code after it is unreachable?}).

The map function can include any number of calls to \tt emit()\rm, which outputs
the resulting intermediate key-value pairs for use by the function defined in the
\tt @reduce \rm section. The types of the values passed to the \tt emit() \rm
function must agree with the signature of the output key-value pair as defined in
the \tt @map \rm type signature. All output pairs from the map function are
subsequently grouped by key by the framework, and passed as input to the \tt
@reduce \rm function.

Currently, the only configuration available is for a file to be passed into the map
function one line at a time, with the line of text being the value, and the
corresponding line number as the key. This requires that the input key/value pair
to the map function is of type \tt (int key‐name,text value‐name)\rm. Extending
this to allow for other input formats is a future goal of the Hog language.

The following is an example of a complete \tt @Map \rm section for a program that
counts the number of times each word appears in a set of files. The map function
receives a single line of text, and for each word in the line (as delineated by
whitespace), it emits the word as the key with a value of one. By emitting the word
as the key, we can allow the framework to group by the word, thus calling the
reduce function for every word.

% section tt_map_rm (end)

\section{\tt @Reduce \rm} % (fold)
\label{sec:tt_reduce_rm}

The reduce function in a MapReduce program takes a list of values that share the
same key, as emitted by the map function, and outputs a smaller set of values to be
associated with another key. The input and output keys do not have to match, though
they often do.

The setup for the reduce section is similar to the map section. However, the input
value for any reduce function is always an iterator over the list of values
associated with its key. The type of the key must be the same as the type of the
key emitted by the map function. The iterator must be an iterator over the type of
the values emitted by the map function.

\begin{verbatim}

@Reduce (key-type key-name, iter<value-type> value-name) -> (key-type, value-type)
{
.
.
.
}
\end{verbatim}

As with the map function, the reduce function can emit as many key/value pairs as
the user would like. Any key/value pair emitted by the reduce function is recorded
in the output file.

Below is a sample at \tt @Reduce \rm section, which continues the word count
example, and follows the @mapsample introduce in the previous section.

% section tt_reduce_rm (end)

\section{\tt @Main \rm} % (fold)
\label{sec:tt_main_rm}

\textbf{Fill this in!} Hog.mapReduce()

% section tt_main_rm (end)

% chapter program_structure (end)

\chapter{Lexical Conventions} % (fold)
\label{cha:lexical_conventions}

\section{Tokens} % (fold)
\label{sec:tokens}

% section tokens (end)

\section{Comments} % (fold)
\label{sec:comments}

Block comments: 

\begin{verbatim}
#{ these are block
   comments }#
\end{verbatim}

Single-line comments are defined to be strings of text included between a '\tt
\#\rm' symbol on the left-hand side an a newline character ('\tt\textbackslash
n\rm') on the right-hand side.

% section comments (end)

\section{Identifiers} % (fold)
\label{sec:identifiers}

% section identifiers (end)

\section{Keywords} % (fold)
\label{sec:keywords}

The reserved words of Hog are a superset of the reserved words of Java, since
Hadoop scripts compile into runnable Java code. The following words are reserved
for use as keywords, and may not be redefined by a programmer:

\begin{multicols}{4}
\tt
\begin{itemize}
  \item[] abstract
  \item[] assert
  \item[] bool
  \item[] boolean
  \item[] break
  \item[] byte
  \item[] case
  \item[] catch
  \item[] char
  \item[] class
  \item[] const
  \item[] continue
  \item[] default
  \item[] do
  \item[] double
  \item[] else
  \item[] enum
  \item[] extends
  \item[] file
  \item[] final
  \item[] finally
  \item[] float
  \item[] for
  \item[] goto
  \item[] hadoop
  \item[] if
  \item[] implements
  \item[] import
  \item[] instanceof
  \item[] int
  \item[] interface
  \item[] iter
  \item[] list
  \item[] long
  \item[] map
  \item[] native
  \item[] new
  \item[] package
  \item[] private
  \item[] protected
  \item[] public
  \item[] real
  \item[] reduce
  \item[] return
  \item[] short
  \item[] static
  \item[] strictfp
  \item[] super
  \item[] switch
  \item[] synchronized
  \item[] text
  \item[] this
  \item[] throw
  \item[] throws
  \item[] transient
  \item[] try
  \item[] void
  \item[] volatile
  \item[] while
\end{itemize}
\rm
\end{multicols}

\textbf{@All: should we not say Java keywords are reserved, and instead rename any
identifiers that could cause name conflicts? Feels weird to have a bunch of
reserved words that are actually semantically meaningless for the language...}

% section keywords (end)

\section{Constants} % (fold)
\label{sec:constants}

% section constants (end)

\section{String Literals} % (fold)
\label{sec:string_literals}

% section string_literals (end)

\section{Variable Scope} % (fold)
\label{sec:variable_scope}

Hog implements what is generally referred to as lexical scoping or block scope. An
identifier is valid within its enclosing block. The identifier is also value for
any block nested within its enclosing block.

% section variable_scope (end)

% chapter lexical_conventions (end)

\chapter{Syntax Notation} % (fold)
\label{cha:syntax_notation}

% chapter syntax_notation (end)

\chapter{Types} % (fold)
\label{cha:types}

\section{Basic Types} % (fold)
\label{sec:basic_types}

The basic types of Hog include \tt int \rm (integer numbers, 64 bytes in size), \tt
real \rm (floating point numbers, 64 bytes in size), \tt bool \rm(boolean values,
true or false) and \tt text \rm (Strings, variable in size). Unlike most languages,
Hog includes no basic character type. Instead, a programmer makes use of \tt
text\rm s of size 1.

\emph{Implementation details} Hog’s primitive types are not so primitive. They are
in fact wrappers around Hadoop classes. For instance, Hog’s \tt int \rm type is a
wrapper around Hadoop's \tt IntWritableclass\rm. The following lists for every
primitive type in Hog the corresponding Hadoop class that the type is built on top
of:

\begin{center}
\begin{tabular}{|c|c|}
    \hline
\textbf{Hog Type} & \textbf{Enclosed Hadoop Class} \\ \hline
\tt int & \tt IntWritable \\ \hline
\tt real & \tt DoubleWritable \\ \hline
\t bool & \tt BooleanWrtiable \\ \hline
\tt text & \tt text\textbf{???}\rm \\ \hline
\end{tabular}
\end{center}

% section basic_types (end)

\section{Derived Types (Collections)} % (fold)
\label{sec:derived_types_collections_}

Derived types include \tt list<T>\rm, \tt set<t>\rm, \tt multiset<t>\rm, and \tt
iter<t>\rm. The \tt list<T> \rm type is an ordered collection of objects of the
same type. The \tt set<T> \rm is an unordered collection of unique objects of the
same type. The \tt multiset<T> \rm is an unordered collection of objects of the
same type, with duplicates allowed. The \tt dict<K,V> \rm is a collection of
key­value pairs, where keys are all of the same type, and values are all of the
same type (keys and values can be of different types from one another). The only
types currently allowed within collections are primitive types, preventing such
constructs as a list of lists. All collections allow for null
entries.\footnote{Note that for \tt set<T>\rm, only one \tt null \rm entry is
allowed, and for \tt map<K,V>\rm, only one \tt null \rm key is allowed.}

% section derived_types_collections_ (end)

\section{Conversions} % (fold)
\label{sec:conversions}

% section conversions (end)

% chapter types (end)

\chapter{Expressions} % (fold)
\label{cha:expressions}

\section{Operators} % (fold)
\label{sec:operators}

\subsection{Arithmetic Operators} % (fold)
\label{sub:arithmetic_operators}

Hog implements all of the standard arithmetic operators. \textbf{Need to say
something about using arithmetic operators with two operands of different type, and
about using them with null values}.

\begin{center}
\begin{tabular}{|c|c|c|c|c|}

\hline \textbf{Operator} & \textbf{Arity} & \textbf{Associativity} &
\textbf{Precedence Level} & \textbf{Behavior} \\ \hline
\tt + \rm & binary & left & 0 & addition \\ \hline
\tt - \rm & binary & left & 0 & minus \\ \hline
\tt * \rm & binary & left & 1 & multiplication \\ \hline
\tt / \rm & binary & left & 1 & division \\ \hline
\tt \% \rm & binary & left & 2 \textbf{??} & mod\footnote{Follows Java's 
\tt \% \rm behavior: a modulus of a negative number is a negative number.} \\ 
\hline
\tt ++ \rm & unary & left & 3 & increment \\ \hline
\tt -- \rm & unary & left & 3 & decrement \\ \hline
\end{tabular}
\end{center}

% subsection arithmetic_operators (end)

\subsection{Logical Operators} % (fold)
\label{sub:logical_operators}

The following are the logical operators implemented in Hog. Note that these
operators only work with two operands of type \tt bool\rm. Attempting to use a
logical operator with an object of any other type results in an internal run-time
exception (see \S \ref{sec:internal_run_time_exceptions}). \textbf{@ALL: should
this be a compile-time exception?}

\begin{center}
\begin{tabular}{|c|c|c|c|c|}

\hline \textbf{Operator} & \textbf{Arity} & \textbf{Associativity} &
\textbf{Precedence Level} & \textbf{Behavior} \\ \hline
\tt or \rm & binary & left & 0 & logical or \\ \hline
\tt and \rm & binary & left & 1 & logical and \\ \hline
\tt not \rm & unary & right & 2 & negation \\ \hline
\end{tabular}
\end{center}

% subsection logical_operators (end)

\subsection{Comparators} % (fold)
\label{sub:comparators}

The following are the comparators implemented in Hog (all are binary operations).
\textbf{Need to say something about comparing two objects of different types, and
null types}.

\begin{center}
\begin{tabular}{|c|c|c|c|}

\hline
\tt < \rm & none & 0 & less than \\ \hline
\tt <= \rm & none & 0 & less than or equal to \\ \hline
\tt > \rm & none & 0 & greater than \\ \hline
\tt >= \rm & none & 0 & greater than or equal to \\ \hline
\tt == \rm & none & 0 & equal \\ \hline
\tt != \rm & none & 0 & not equal \\ \hline

\end{tabular}
\end{center}

% subsection comparators (end)

\subsection{Assignment} % (fold)
\label{sub:assignment}

There is one single assignment operator, '\tt =\rm'. Expressions involving the
assignment operator have the following form:

\begin{verbatim}
lvalue = expression | PRIMITIVE | DERIVED
\end{verbatim}

At compile-time, the compiler checks that both the result of the \tt expression \rm
(or \tt PRIMITIVE \rm or \tt DERIVED\rm) and \tt lvalue \rm have the same type. If
not, the compile throws a \tt TypeMistmatchException\rm. \textbf{@ALL: should this
be a run-time exception?}

% subsection assignment (end)

% section operators (end)

% chapter expressions (end)

\chapter{Declarations} % (fold)
\label{cha:declarations}

While it is not specified in the grammar of Hog, like many other programming
languages, a user is only allowed to use variables/functions after they have been
declared. \textbf{Another sentence of introduction}.

\section{Type Specifiers} % (fold)
\label{sec:type_specifiers}

Every variable, be it a \tt primitive-type \rm or a \tt derived-type \rm has to be
assigned a type upon declaration, for instance,

\begin{verbatim}
list<int> myList
\end{verbatim}

Declares the variable \tt myList \rm to be a \tt list \rm of \tt int\rm s. And,

\begin{verbatim}
text myText
\end{verbatim}

Declares the variable \tt myText \rm to be of type \tt text\rm .

% section type_specifiers (end)

\section{Declarations} % (fold)
\label{sec:declarations}

\subsection{Null Declarations} % (fold)
\label{sub:null_declarations}

If a variable is declared but not initialized, the variable becomes a
\textbf{\emph{null reference}}, which means it points to nothing, holds no data,
and will fail any comparison (see \S \ref{sec:operators}) for a discussion of how
\tt null \rm affects comparisons and elementary arithmetic and boolean operations).

% subsection null_declarations (end)

\subsection{Primitive-type Variable Declarations} % (fold)
\label{sub:primitive_type_variable_declarations}

Variables of one of the primitive types, including \tt int\rm, \tt real\rm, \tt
text \rm or \tt bool \rm are declared using the following patterns:

\begin{enumerate}
  \item \tt variable-type variable-name \rm $\hfill$ (a null declaration)
  \item \tt variable-type variable-name = primitive-expr \rm $\hfill$ (declaration 
with initialization)
\end{enumerate}

When the first pattern is used, we say that the variable is
\textbf{\emph{uninitialized}}, and has the value \tt null\rm. When the second
pattern is used, we say that the variable is \textbf{\emph{initialized}}, and has
the same value as the value of the result of the \tt
primitive-assignment-expr\rm. The \tt primitive-assignment­expr \rm must return
a value of the right type, or the compiler will fail citing a syntax error. The \tt
primitive­assignment­expr \rm may contain an expression involving both other
variables and unnamed raw primitives (e.g. 1 or 2), an expression involving only
other variables or unnamed raw primitives, or a single variable, or a single
unnamed raw primitive.

% subsection primitive_type_variable_declarations (end)

\subsection{Derived-Type Variable Declarations} % (fold)
\label{sub:derived_type_variable_declarations}

Derived-type variables are declared using the following patterns:

\begin{enumerate}
  \item \tt variable-type variable-name \rm
  \item \tt variable-type variable-name = derived-expr \rm
  \item \tt variable-type variable-name(parameter-list) \rm \\
  \tt parameter-list -> parameter, parameter-list | parameter \rm
\end{enumerate}

The first two patterns operate in essentially the same way as for primitive­type
variables. When the first pattern is used, we say that the variable is
\textbf{\emph{uninitialized}}, and has the value \tt null\rm. If a user attempts to
use any type­specific operations (for instance, \tt size(myList)) \rm on an
uninitialized variable, the program will through a run­time exception (see \S
\ref{cha:exception_handling} for a discussion of exceptions). When the second
pattern is used, the variable is \textbf{\emph{initialized}} to the result of the
\tt derived­expr\rm.

Because derived­type variables often have additional structure that needs to be
defined at initialization, a third pattern is provided. In this pattern, the user
can specify a list of \textbf{\emph{parameters}} to initialize the object. For
instance,

\begin{verbatim}
list<int> myList(5)
\end{verbatim}

Specifies that \tt myList \rm should be initialized with five \tt null \rm values.

% subsection derived_type_variable_declarations (end)

\subsection{Function Declarations} % (fold)
\label{sub:function_declarations}

See \S \ref{sec:tt_functionsrm}.

% subsection function_declarations (end)

% section declarations (end)

\section{Initialization} % (fold)
\label{sec:initialization}

% section initialization (end)

% chapter declarations (end)

\chapter{Statements} % (fold)
\label{cha:statements}

\section{Expression Statement} % (fold)
\label{sec:expression_statement}

% section expression_statement (end)

\section{Compound Statement} % (fold)
\label{sec:compound_statement}

% section compound_statement (end)

\section{Flow-Of-Control Statements} % (fold)
\label{sec:flow_of_control_statements}

\textbf{@Paul: think about what you want here! Maybe include example for each?} The
following are the \textbf{\emph{flow-of-control}} statements included in Hog:

\begin{itemize}
  \item[] \tt if (expression) statement \rm
  \item[] \tt if (expression) statement else statement \rm
  \item[] \tt if (expression) statement elif (expression statement) .. else 
  statement \rm
  \item[] \tt switch(expression) statement \rm
\end{itemize}

In the above statements, the ‘...’ signifies an unlimited number of elseif
statements, since there is no limit on the number of elseif statements that can
appear before the final else statement. In all forms of the if statement, the
expression will be evaluated as a Boolean. If the expression is a number, then any
non­zero number will be considered true and zero will be treated as false. In the
second statement above, when the expression in the if statement evaluates to false,
then the else statement will execute. In the third statement above with if, elseif
and else statements, the statement will be executed that follows the first
expression evaluating to true. If none of these expressions evaluate to true, then
the else statement is executed.

The switch statement causes control to transfer to a statement depending on the
matching case label. There can be an unlimited number of case labels within the
switch statement, so that the switch will operate as such:

\begin{verbatim}
switch(expression) {
  case consant-expression : statment
  default : statement
}
\end{verbatim}

An expression is passed in to the switch statement and then the flow of control
will fall through the switch and the expression will then be compared to the
constant expression next to each case label. When the switch expression matches the
expression next to a specific case, the statement for that case is executed. If the
flow of control falls to the bottom of the switch without finding an equality, then
the default statement will be executed. The case constants are converted to the
switch expression type. There cannot be two case expressions with the same value
after conversion. In addition, there can only be one default label within each
switch.

The above control statements can all be nested within each other.

% section flow_of_control_statements (end)

\section{Iteration Statements} % (fold)
\label{sec:iteration_statements}

Iteration statements signify looping and can appear in one of the two following
forms:

\begin{itemize}
  \item[] \tt while ( expression ) statement-list \rm
  \item[] \tt for (expression ; expression ; expression ) statement-list \rm
  \item[] \tt foreach expression in iterable-object statement-list \rm
\end{itemize}

The following statement,

\begin{verbatim}
expression1
while(expression2) {
  statement
  expression3
}
\end{verbatim}

is equivalent to,

\begin{verbatim}
for (expression1; expression2; expression3) statement
\end{verbatim}

\textbf{@Ben: clean up the above, needs to be a specific example, because it's not
always true in this generic a fashion}.

In the above while statement, expression1 will typically represent the
initialization of a variable, then at each iteration through the while loop, the
current Boolean value of expression2 is evaluated. Expression2 will be a test or
condition. If expression2 evaluates to true, then iteration will continue through
the while loop. Expression3 normally represents an increment step and its value
changes at each step through the while. Expression3 could be a part of the
condition tested in expression2. The first time expression2 evaluates to false,
iteration through the loop ends and drops to the code that comes after the closing
bracket of the while statement.

In the above for statement, expression1 is the initialization step, expression2 is
the test or condition and expression3 is the increment step. At each step through
the for loop, expression2 is evaluated. When expression2 evaluates to false,
iteration through the loop ends.

When the foreach starts to execute, the iteration starts at the first element in
the array or list and the statement executes during every iteration. The iteration
ends when the statement has been executed for each item in the array (or list) and
there are no items left to iterate through.

\subsection{Example 1} % (fold)
\label{sub:example_1}

\begin{verbatim}
int i = 0
while (i < 10) {
  print(i)
  i++
}
\end{verbatim}

% subsection example_1 (end)

\subsection{Example 2} % (fold)
\label{sub:example_2}

\begin{verbatim}
for (int i = 0; i < 10; i++) {
  print(i)
}
\end{verbatim}

% subsection example_2 (end)

\subsection{Example 3} % (fold)
\label{sub:example_3}

\begin{verbatim}
list<int> iList()
iList = [0,1,2,3,4,5,6,7,8,9]
foreach i in iList {
  print(i)
}
\end{verbatim}

% subsection example_3 (end)

% section iteration_statements (end)

% chapter statements (end)

\chapter{Built-in Functions} % (fold)
\label{cha:built_in_functions}

\textbf{Overview of built-in functions? How they are called on objects...}

\section{System-level Built-ins} % (fold)
\label{sec:system_level_built_ins}

Hog includes a number of system­level built­in functions that can be called from
various sections of a Hog program. The functions are:

\begin{itemize} 

\item[] \tt void emit(key, value) \rm \\

This function can be called from the \tt @Map \rm and \tt @Reduce \rm sections in
order to communicate the results of the map and reduce functions to the Hadoop
platform. The types of the key/value pairs must match those defined as the output
types in the header of each section.

\item[] \tt void mapReduce() \rm \\

This function can be called from the \tt @Main \rm section in order to initiate the
mapreduce job, as definied in the \tt @Map \rm and \tt @Reduce \rm sections. Any
Hog program that implements mapreduce will need to call this function in \tt
@Main\rm.

\item[] \tt void print(toPrint) \rm \\

This function can be called from the \tt @Main \rm section in order to print to
standard output. The argument must be a primitive type.

\end{itemize}

% section system_level_built_ins (end)

\section{Object-level Built-ins} % (fold)
\label{sec:object_level_built_ins}

\textbf{@ALL:} introduction.

\subsection{List} % (fold)
\label{sub:list}

\begin{itemize}

\item \tt void sort() \rm \\

Destructive?? function that sorts the items in the list in ascending order.

\item \tt int len() \rm \\

Returns an int with the number of elements in the list.

\item \tt void append(itemToAppend) \rm \\

Appends the object passed to the end of the list. The object must be of the same
type as the list, or the operation will result in a \textbf{compile-time or
run-time} exception.

\item \tt <type> get(int index) \rm \\

Returns the item from the list at the specified index.

\end{itemize}


% subsection list (end)

\subsection{text} % (fold)
\label{sub:text}

The following function can be called on a \tt text \rm object:

\begin{itemize}

\item[] \tt list<text> tokenize(text fullText, text delimiter) \rm \\

\tt tokenize() \rm can be called on a \tt text \rm object to tokenize it into a
list of \tt text \rm objects based on the delimiter. The delimiter is not included
in any of the \tt text \rm objects in the returned list.

\end{itemize}

% subsection text (end)

% section object_level_built_ins (end)

% chapter built_in_functions (end)

\chapter{System Configuration} % (fold)
\label{cha:system_configuration}

\textbf{What facts about Hadoop instance does Hog need, and how to do you define
them in a Hog program. Are there path variables that Hog depends on?}

The user must set configuration variables in the \tt hog.rb \rm build script to
allow the Hog compiler to link the Hog program with the necessary jar files to run
the MapReduce job. The user must also specify the job name within the Hog source
file.

\begin{itemize}

\item[] HADOOP\_HOME: absolute path of hadoop folder
\item[] HADOOP\_VERSION: hadoop version number
\item[] JAVA\_HOME: absolute path of java executable
\item[] JAVAC\_HOME: absolute path of javac executable
\item[] HOST: where to job is rsynced to and run
\item[] LOCALMEM: how much memory for java to use when running in local mode 
\item[] REDUCERS: the number of reduce tasks to run, set to zero for map­only jobs

\end{itemize}

% chapter system_configuration (end)

\chapter{Parsing Tools} % (fold)
\label{cha:parsing_tools}

\textbf{INCLUDE GRAPHIC}

JFlex is a lexical analyzer generator for Java, and written in Java. JFlex is
designed to work together with the LALR parser generator CUP.

% chapter parsing_tools (end)

\chapter{Linkage and I/O} % (fold)
\label{cha:linkage_and_i_o}

\textbf{INCLUDE GRAPHIC}

\section{Usage} % (fold)
\label{sec:usage}

% section usage (end)

To build and run a Hog source file there is an executable Ruby script \tt hog.rb
\rm that automates the compilation and linking steps for the user.

Usage: \tt hog.rb [--hdfs|--local] job <job args> \rm
\begin{itemize}
  \item[] \tt --hdfs\rm: if job ends in '.hog' or '.java' and the file exists, link it against the hadoop JARFILE and then run it on HOST.
  \item[] \tt --local\rm: run on local host.
\end{itemize}

\section{Example} % (fold)
\label{sec:example}

\tt hog.rb ­­local WordCountJob.hog ­­input someInputFile.txt ­­output
./someOutputFile.tsv \rm

This runs the \tt wordCount \rm job in \emph{local} mode (i.e. not on a Hadoop
cluster).

% section example (end)

% chapter linkage_and_i_o (end)

\chapter{Exception Handling} % (fold)
\label{cha:exception_handling}

Similar to other programming languages (Java, C++), Hog uses an exception model in
which an exception is thrown and can be caught by a catch block. Code should be
surrounded by a try block and then any exceptions occurring within the try block
will subsequently be caught by the catch block. Each try block should be
associated with at least one catch block. However, there can be multiple catch
blocks to handle specific types of exceptions. In addition, an optional finally
block can be added. The finally block will execute in all circumstances, whether
or not an exception is thrown. The structure of exception handling should be
similar to this, although there can be multiple catch blocks and the finally block
is optional:

\begin{verbatim}
try {
  statement-list
} catch (exception) {
  statement-list
} finally {
  statement-list
}
\end{verbatim}

Because the proper behavior of a Hog program is dependent on resources outside of
the language (i.e. the proper behavior of the user’s Hadoop software), there are
more sources exceptions in Hog than most general purpose languages. These sources
can be divided into three categories: \textbf{\emph{compile­time exceptions}},
\textbf{\emph{internal run­time exceptions}} and \textbf{\emph{external run­time
exceptions}}.

\section{Compile-time Exceptions} % (fold)
\label{sec:compile_time_exceptions}

The primary cause of most compile­time exceptions in Hog are syntax errors. Such
errors are unrecoverable because it is impossible for the compiler to properly
interpret the user program. Some compilers for other languages offer a limited
amount of compile­time error correction. Because Hog programs are often designed
to process gigabytes or terabytes of data at a time, the standard Hog compiler
offers no compile­time error correction. The assumption is that a user would
rather re­tool their program than risk the chance of discovering, only after hours
of processing, that the compilers has incorrectly assumed what the user meant. The
following are Hog compile­time exceptions:

\textbf{Add descriptions for each exception}

\begin{itemize}
  \item[] ProgramStructureException
  \item[] NoSuchVariableException
  \item[] NoSuchMethodException
  \item[] UnsupportedOperationException
  \item[] RedundantDeclarationException
\end{itemize}

% section compile_time_exceptions (end)

\section{Internal Run-time Exceptions} % (fold)
\label{sec:internal_run_time_exceptions}

Internal run­time exceptions include such problems as I/O exceptions (i.e. a
specified file is not found on either the user’s local file system or the
associated Hadoop file system), type mismatch exceptions (i.e. a program attempts
to place two elements of different types into the same list) and parsing
exceptions. \textbf{Another sentence or two}. The following are Hog internal
rum­time exceptions:

\textbf{Add descriptions for each exception}

\begin{itemize}
  \item[] FileNotFoundException
  \item[] FileLoadException
  \item[] ArrayOutOfBoundsException
  \item[] IncorrectArgumentException
  \item[] TypeMismatchException
  \item[] HogMapFunctionException
  \item[] HogReduceFunctionException
  \item[] NullPointerException
  \item[] NumberFormatException
  \item[] ArithmeticException
  \item[] IOException
  \item[] InterruptedException
  \item[] ParseException
\end{itemize}

% section internal_run_time_exceptions (end)

\section{External Run-time Exceptions} % (fold)
\label{sec:external_run_time_exceptions}

Hog programs are primarily meant to be run on a Hadoop cluster, and therefore a
third source of exceptions for the language comes from errors associated with the
cluster. If the cluster becomes corrupted in the middle of processing a Hog script
(\textbf{can we say anything but ``the hog script dies?'' @Kurry­­­does Hadoop
have a single point­of­failure? If so, we should say something about it here}).
The following are external run­time exceptions (\textbf{@kurry, any input on
these?}):

\textbf{Add descriptions for each exception}

\begin{itemize}
  \item[] HadoopCompilationException
  \item[] HadoopFailuerException
  \item[] HadoopExecutionException
  \item[] HadoopMapFunctionException
  \item[] HadoopReduceFunctionException
\end{itemize}

% section external_run_time_exceptions (end)

% chapter exception_handling (end)

\chapter{Grammar} % (fold)
\label{cha:grammar}

\begin{verbatim}

if ( expression ) statement
if ( expression ) statement else statement
if ( expression ) statement elseif ( expression ) statement... else statement

switch ( expression ) statement if ( expression ) statement
if ( expression ) statement else statement
if ( expression ) statement elseif ( expression ) statement... else statement

switch ( expression ) statement

while ( expression ) statement
for ( expression; expression; expression ) statement
foreach ( expression in array or list) statement
@Functions
return­type function­name­1 (parameter­list) {
exprlist }
return­type function­name­2 (parameter­list) { exprlist
}
@Map: (type key­name, type value­name) ­> (type, type) {
.
.
.
}


\end{verbatim}

% chapter grammar (end)


\end{document}
