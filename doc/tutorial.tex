\documentclass{article} \usepackage{fancyhdr, multicol}

\title{\huge \textbf{Hog Language Tutorial}}

\begin{document}
\maketitle
\large
\begin{itemize}
  \item[] \textbf{Jason Halpern} $\hfill$ Testing/Validation
  \item[] \textbf{Samuel Messing} $\hfill$ Project Manager
  \item[] \textbf{Benjamin Rapaport} $\hfill$ System Architect
  \item[] \textbf{Kurry Tran} $\hfill$ System Integrator
  \item[] \textbf{Paul Tylkin} $\hfill$ Language Guru
\end{itemize}
\normalsize
\newpage

\section*{Introduction}
\label{sec:introduction}

Hog gives users with some programming experience a gentle introduction to
MapReduce, a popular programming model for distributed computation. In a Hog
program, a user specifies an \tt @Map \rm function, which operates on key-value
pairs (read from a text file), and outputs intermediate key-value pairs. The user
also specifies an \tt @Reduce \rm function, which groups the intermediate key-value
pairs by key, and outputs a final set of key-value pairs. This model of computation
has been widely adopted for distributing large computations that might be
considered "embarassingly parallelizable."

\subsection*{Program Structure} % (fold)
\label{sub:program_structure}
Every Hog program has four sections, defined in the following order:
\begin{description}
\item[\tt @Functions\rm:] An optional section which defines functions used throughout the program.
\item[\tt @Map\rm:] This section defines the map function that takes the input key-value pairs and outputs intermediate key-value pairs.
\item[\tt @Reduce\rm:] This section defines the reduce function that takes a single key from the set of intermediate key-value pairs output by the map function, and all of the values associated with that key, and reduces them to a final output.
\item [\tt @Main\rm:] The entry point for the program which initiates the MapReduce routine and can perform other local (non-distributed) computations.
\end{description}

\section*{Word Count} 
\label{word_count} 

Let's assume we have thousands of large text files, and we would like to get a
cross-file word count for each word that appears in any of the files. We also have
a cluster of computers to help us complete this task. The following short Hog
program will produce a single output file with each word and its associated count.

\subsection*{Word Count Code}
\begin{verbatim}
      
      @Map (int lineNum, text line)  ->  (text, int) {
            foreach word in line.tokenize(" ") {
                  emit(word, 1)
            }
      }
      
      @Reduce (text word, iter<int> values) -> (text, int) {
            int count = 0
            while (values.hasNext()) {
                  count = count + values.next()
            }
            emit(word, count)
      }
      
      @Main {
            mapReduce()
      }
      
\end{verbatim}

\subsection*{Running the Word Count Code} % (fold)
\label{sub:running_the_word_count_code}

To run the program \tt WordCount.hog\rm, the user types the following into the
terminal: \\

\noindent \tt hog --hdfs WordCount.hog --input inputFile.txt
--output wordCounts.csv \rm \\

Here, \tt --hdfs \rm indicates that we want to run the job on the user's Hadoop
cluster (specified in a configuration file, see the Hog reference manual for
details on how to properly orient Hog to your Hadoop cluster). If, alternatively,
we wanted to run the job locally, we can type \tt --local\rm. The second parameter
is the name of the program, here \tt WordCount.hog\rm. The last two parameters
indicate the input file (or directory, in which case every file in the directory
will be used as input in turn) and the desired name of the output file.

% subsection running_the_word_count_code (end)

\subsection*{Word Count Explanation}

The general idea of this program is that we want to read every line of text from
every file, and then, grouping by word, output the total number of times we
encountered each word. Since we want to group by word, we will use the words
themselves as the intermediate key output by the \tt @Map \rm function. This will
allow us to group each word's value and send them all together in one key-value
pair to the \tt @Reduce \rm function.

\subsubsection*{\tt @Functions \rm}

The first thing we notice is that this program does not contain an \tt @Functions
\rm section. This section is optional, and only needs to be included if the user
wants to write his or her own subroutines to be used elsewhere in the program.

\subsubsection*{\tt @Map \rm}

This section's job is to read in a line of text from a file, and simply output each
word as the key with a value that indicates we have just encountered it. We will
use this value later to perform the summation.

The first line of this section is the \tt @Map \rm header, which defines the
\textbf{\emph{signature}} of the \tt @Map \rm function. In the current release, all
Hog programs read input files one line at a time, where the file offset of the line
is the key, and the text of the line is the value. \emph{This means that for all
Hog programs, the only allowable types for the input key-value pair is \tt (int\rm,
\tt text)\rm}. The inputs are also \textbf{\emph{named}} in the signature in order
to reference them in the body of the function.

The input signature is followed by an arrow, followed by the type signature of the
outputted intermediate key-value pairs. In this case, we will output each word as
\tt text \rm and its count as an \tt int\rm. These values are
\textbf{\emph{unnamed}}, as they cannot be referenced in the \tt @Map \rm section.

The \tt int \rm type represents an \textbf{\emph{integer number}} such as 0, 1, -2,
3, 5, etc. In addition, Hog has the type \tt real \rm which represents
\textbf{\emph{real numbers}} such as 0.1, 2.141, etc. The \tt text \rm type is
Hog's string type, and represents a sequence of characters. To create a \tt text
\rm object, simply include a string of characters between two double quotes (e.g.
\tt "hello world 123"\rm).

In the body of the function, we split the line of text passed in as the value into
words delineated by whitespace by using the built-in function \tt tokenize()\rm. We
then iterate through the \tt list \rm of words (of type \tt list<text>\rm) that \tt
tokenize() \rm returns using a \tt foreach \rm loop. Notice that you call \tt
tokenize() \rm "on" a \tt text \rm object. \tt text \rm objects are the only type
of objects that support this function. Attempting to call the function on an object
of a different type (e.g. \tt count.tokenize() \rm for the variable \tt count \rm
in this example) would lead to an error, called an \textbf{\emph{exception}}.
Exception handling is outside of the scope of this tutorial. Please see the
language reference manual for guidance on how to anticipate and handle exceptions.

In the body of the \tt foreach \rm loop, we use the built-in function \tt emit()
\rm to output a key-value pair, which the framework then groups by key when passing
to the \tt @reduce \rm section. In this case, since we want to group by the word
itself, we emit the word and the value \tt 1\rm, which we will later use to
calculate the totals in the \tt @Reduce \rm section.

\subsubsection*{\tt @Reduce \rm}


In this section, for each word (the key) emitted by the \tt @Map \rm section, we
will simply add up all the counts (the values) emitted for each particular word to
get the final count. It should now be clear why we emitted the value \tt 1 \rm for
each word in the \tt @Map \rm section, as we do so once for every instance of
seeing a particular word.

Since the inputs to this section are grouped by key, \tt @Reduce \rm will receive a
word and an \textbf{\emph{iterator}} (referred to as an \tt iter \rm in Hog) over
all of that word's values (the \tt 1\rm's we emitted in the \tt @Map \rm section).
For \emph{every} word, this function will receive an iterator over all of the
values emitted by the \tt @Map \rm function for \emph{that} word. This is why the
header for this section has the word as the key and an iterator over a \tt list \rm
of \tt int\rm s as the value. The key type of the input to the reduce function
\emph{must match} the key type of the output of the map function. Similarly, the
values type of the reduce function is \emph{always} an iterator over the type of
the value output by the map function.

Since we want to output a word and its associated word count, \tt @Reduce \rm will
output \tt text \rm and \tt int \rm for each word.

In the body, we initialize the \textbf{\emph{variable}} \tt count \rm to \tt 0\rm,
and then iterate through the list of values using a familiar \tt while \rm loop,
adding each value of \tt 1 \rm to a running total (recall that \tt count \rm has
type \tt int\rm, which means it can represent an integer value). To do this, we use
the built-in functions on iterators \tt hasNext()\rm---which returns \tt true \rm
if the iterator contains more values and \tt false \rm otherwise---and \tt
next()\rm---which returns the next value in the \tt list \rm and moves the
iterator position forward. The statements inside the \tt while \rm loop continue to
execute until we have seen every variable in the \tt iter \rm object (when \tt
values.hasNext() \rm evalutes to \tt false\rm).

After we have a full count for the input word, we emit the word and its total count
as our final output. The framework then takes care of writing these emitted
key-value pairs to an output file, the name of which is specified by the user when
the program is above (see the section above, \textbf{Running the Word Count Code}).

\subsubsection*{@Main}

In this section, we simply call the built-in function \tt mapReduce()\rm, which
initiates the MapReduce program as specified by the previous sections and the
command line arguments.

\subsubsection*{Sample Output}

Here is the first fifty lines of output generated by \tt WordCount.hog \rm when run
on a single file containing the text of \emph{War and Peace}:

\begin{verbatim}
  31784, the
  21049, and
  16389, to
  14895, of
  10056, a
  8314, in
  7847, he
  7645, his
  7425, that
  7255, was
  5540, with
  5316, had
  4492, not
  4209, at
  4162, her
  4009, I
  3757, it
  3744, as
  3495, on
  3488, him
  3308, for
  3134, is
  2888, but
  2762, The
  2718, you
  2636, said
  2620, she
  2526, from
  2390, all
  2387, were
  2354, be
  2333, by
  2031, who
  2006, which
  1910, have
  1812, He
  1777, one
  1727, they
  1693, this
  1645, what
  1566, or
  1561, an
  1554, Prince
  1550, so
  1541, Pierre
  1466, been
  1439, did
  1424, up
  1409, their
  1342, would
\end{verbatim}

\section*{MergeSort} \label{merge_sort} 

In this example, we will sort numbers in text files using a version of the
divide-and-conquer algorithm MergeSort. We will assume that our text files contain
lines of integers, delimited by commas. The idea is for each call to map to sort a
small list of numbers on a single line of text, and for reduce to merge all of the
sorted lists it receives.

\subsection*{MergeSort Code}
\begin{verbatim}

      @Functions: {

        #{ merge: Takes two sorted lists and merges them into
                one combined and properly sorted list. }#
        list<int> merge(list<int> sortedList1, list<int> sortedList2) {

              list<int> mergedList()

              # pointers to next value of each sorted list
              int ind1 = 0
              int ind2 = 0

              # merge all values while neither list is empty
              while( ind1 < sortedList1.size() and ind2 < sortedList2.size() ) {

                    # insert the smaller of the 2 values and update index pointers
                    if(sortedList1.get(ind1) < sortedList2.get(ind2)) {
                          mergedList.add(sortedList1.get(ind1++))
                    }
                    else { 
                          mergedList.add(sortedList2.get(ind2++))
                    }
              }

              # insert any remaining elements from sortedList1
              while (ind1 < sortedList1.size()) {
                    mergedList.add(sortedList1.get(ind1++))
              }

              # insert any remaining elements from sortedList2
              while (ind2 < sortedList2.size()) {
                    mergedList.add(sortedList2.get(ind2++))
              }

              return mergedList
        }
      }
      
      @Map (int lineNum, text line) -> (text, list<int>) {

            text reduceKey = "reduceKey"
            list<int> sortedInts()
            
            # put every number from line into list
            foreach number in line.tokenize(",") {
                  sortedInts.add((int) number)
            }
            
            # sort list
            sortedInts.sort()
            
            # for every line of numbers, emit the sorted ints with an identical key
            emit(reduceKey, sortedInts)
      
      }
      
      # reduce will get a list of sorted lists, and merge them 2 at a time
      @Reduce (text key, iter<list<int>> allSortedLists) -> (text, list<int>) {
            
            # only one output key
            text reduceKey = ""
            
            # begin with the first list as the fully sorted list 
            list<int> allSortedNums = allSortedLists.next()
            
            #{ merge the allSortedNums with the next 
               sorted list until all lists have been merged }#
            while(allSortedLists.hasNext()) {
                  allSortedNums = merge(allSortedNums, allSortedLists.next())
            }
            
            emit(reduceKey, allSortedNums)
            
      }
      
      @Main {
            print("Beginning sort.\n")
            mapReduce()
            print("Sort complete.\n")
      }

\end{verbatim}

\subsubsection*{\tt @Functions \rm}

In this section, we define a function called \tt merge\rm, which takes two sorted
\tt list\rm s of \tt int\rm s, and returns a merged \tt list \rm of the two in
sorted order. The way to define a function should be familiar to programmers
comfortable with C or Java. In the first line of the body of the function, we are
creating a new, empty list. Following that, we demonstrate a few flow of control
statements such as \tt while \rm loops, and \tt if else \rm statements, the \tt and
\rm boolean operator, and comparators all of which should also be familiar.

The function works as follows: it starts with cursors at the beginning of both
input lists. It continues to add elements from the first list to the combined list
until there is a smaller element in the second list, at which point it starts
adding elements from that list instead. It continues switching back and forth
between the two lists in this manner to ensure that it adds all the elements from
both lists in the proper order. It finally checks that no elements in either list
has been left out, and then returns the new, combined list.

Also included in this section are some built-in list functions, such as \tt
.size()\rm, to get the number of elements in a list, \tt .add()\rm, to add an
element onto the end of the list, and \tt .get() \rm to get an element at a
specific index in the list.

\subsubsection*{\tt @Map \rm} 

The map function reads in a line of comma-separated integers as \tt text\rm, and
outputs a list of the integers in sorted order. To do this, we introduce
\textbf{\emph{casting}}, which allows enables transforming a value of one type into
another type. In order to cast, the programmer must put the type he or she wants to
cast to in parenthesis before the value or variable name. In this case, we are
casting a \tt text \rm to an \tt int\rm, which is a very common operation in Hog,
since all input is read in as \tt text\rm. If the string contained in the \tt text
\rm object is not a valid number (e.g. \tt "1ab2"\rm), the cast will result in an
exception. A more robust version of the MergeSort program would include some
exception handling to handle these cases. It's worth noting that the current code
is somewhat brittle, in that if such a string of characters is encountered, the
program will fail.

To sort the list of \tt int\rm s that have been read in from the input, we call a
built-in function on lists called \tt.sort()\rm. This function sorts the elements
of a \tt list \rm in ascending order according to their lexicographic ordering. The
\tt sort \rm method is well-defined in part because \tt list \rm objects can only
contain elements of the same type, and only supports primitive types. See the
language reference manual for more details about \tt lists \rm and their built-in
functions.

Finally, we emit the sorted list as a value, with identical keys for each list, so
that they are all sent to the same reducer. This is a common trick employed in Hog
when we only care about one value instead of two (i.e. a key-value pair).

\subsubsection*{\tt @Reduce \rm}

The reduce function receives an iterator over \emph{all} of the sorted lists from
the map function, and merges them together one by one using the \tt merge() \rm
function we defined earlier.

\subsubsection*{\tt @Main \rm}

In this section, we demonstrate that arbitrary code can be performed locally in the
\tt @Main \rm block. While the \tt @Main \rm must always call the \tt mapReduce()
\rm function to begin the map reduce program, it can also perform locally any code
that could be written in a function. In this example, we use the built-in function
\tt print() \rm to print messages to standard out and let the user know that the
MapReduce job has completed.

\subsubsection*{Sample Output}

Because we used the same, empty \tt text \rm ( \tt text reduceKey = "" \rm) for the
keys for all key-value pairs sent to the \tt @Reduce \rm section, we end up with a
sorted list of numbers all on one line. An example is given below,

\begin{verbatim}
  ,-123,-1,0,0,0,1,1,2,3,4,9,15,1234,6234,53854,123495,82737473745
\end{verbatim}

\noindent Notice that the string begins with a space and then a comma, this is
because of the empty \tt text \rm key. If we had instead put some non-empty \tt
text \rm for the key, it would've shown up here.

\end{document}
